Languages feel tidy until they don’t. Why can we say \emph{Colorless green ideas sleep furiously} but not *\emph{I’ve finished it yesterday}? Why does a tongue‑twister like \emph{The rat the cat the dog chased killed ate the cheese} sound wrong, even though every grammar book says it is fine? I argue that grammar lives in communities, not inside an abstract faculty. A sentence is grammatical when the community keeps using a particular form to package a particular meaning and nobody blocks it. Five forces decide the fate of each sentence: the form–meaning link itself, whether that meaning fits the situation, how hard the sentence is to process word‑by‑word, how firmly the community has entrenched it, and whether a never‑ever rule bars it outright. Crunch those forces and you get two scores: one measures objective grammaticality, the other predicts how bad the sentence feels in your gut. Centre embeddings look fine on paper but hit your working memory, so you shudder; the infamous comparative illusion sneaks past your first check because it maps poorly to meaning yet hides the mismatch. This simple calculus also predicts language change: once enough speakers find a new pattern useful and easy enough to process, it shoots up in an S‑curve—the same shape you see in everything from viral videos to epidemics. Grammaticality, it turns out, is just another social contagion, curbed or boosted by concrete pressures we can measure. My talk shows the math, the psycholinguistic evidence, and why this community‑anchored view dissolves long‑standing puzzles about errors that stick and errors that vanish.