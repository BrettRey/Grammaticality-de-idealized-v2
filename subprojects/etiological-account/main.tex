% !TEX TS-program = xelatex
\documentclass[12pt,letterpaper]{article}

% ===========================
% BASIC PACKAGES
% ===========================
\usepackage[british]{babel}
\usepackage[final]{microtype}
\usepackage{amsmath,amssymb}
\numberwithin{equation}{section}
\usepackage{graphicx}
\usepackage{booktabs}
\usepackage{array}
\usepackage{float}
\usepackage{xspace}
\usepackage{tikz}
\usetikzlibrary{calc,positioning,arrows.meta,fit,backgrounds,shapes.geometric}
\usepackage{fontspec}
\setmainfont{EB Garamond}[Numbers=OldStyle,Ligatures=TeX]
\newfontfamily\ipafont{Charis SIL}
\newcommand{\ipa}[1]{{\ipafont #1}}
\providecommand{\liningnums}[1]{{\addfontfeatures{Numbers=Lining}#1}}
\setmonofont{Inconsolata}[Scale=MatchLowercase]

% ===========================
% PAGE LAYOUT
% ===========================
\usepackage[
  letterpaper,
  inner=1.25in,
  outer=1in,
  top=1in,
  bottom=1.25in,
  marginparwidth=0.6in,
]{geometry}

% ===========================
% HEADINGS
% ===========================
\usepackage{titlesec}
\titleformat{\section}{\normalfont\scshape}{\llap{\thesection\quad}}{0pt}{}
\titleformat{\subsection}{\normalfont\scshape}{\thesubsection\quad}{0pt}{}
\titleformat{\subsubsection}{\normalfont}{\thesubsubsection\quad}{0pt}{}
\titleformat{\paragraph}[runin]{\normalfont\scshape}{\theparagraph}{1em}{}
\titlespacing*{\section}{0pt}{2ex plus 1ex minus .2ex}{1ex plus .2ex}
\titlespacing*{\subsection}{0pt}{1.5ex plus 1ex minus .2ex}{0.5ex plus .2ex}
\titlespacing*{\subsubsection}{0pt}{1ex plus 0.5ex minus .1ex}{0.3ex plus .1ex}
\titlespacing*{\paragraph}{0pt}{1ex plus 0.5ex minus .1ex}{1em}

% ===========================
% RUNNING HEADS
% ===========================
\usepackage{fancyhdr}
\pagestyle{fancy}
\fancyhf{}
\fancyhead[L]{\small\scshape\leftmark}
\fancyhead[R]{\small\thepage}
\setlength{\headheight}{26pt}
\addtolength{\topmargin}{-12.4pt}
\renewcommand{\headrulewidth}{0pt}

% ===========================
% COLOURS & HYPERLINKS
% ===========================
\usepackage{xcolor}
\definecolor{linkmaroon}{RGB}{128,0,32}
\usepackage{hyperref}
\hypersetup{
  colorlinks=true,
  linkcolor=linkmaroon,
  citecolor=linkmaroon,
  urlcolor=linkmaroon,
  pdftitle={Grammaticality de-idealized: An etiological account},
  pdfauthor={Brett Reynolds},
}

% ===========================
% QUOTATIONS
% ===========================
\usepackage[style=american]{csquotes}

% ===========================
% SEMANTIC MACROS
% ===========================
\newcommand{\term}[1]{\textsc{#1}}
\newcommand{\mention}[1]{\textit{#1}}
\newcommand{\mentionh}[1]{⟨#1⟩}
\newcommand{\olang}[1]{\textit{#1}}
\newcommand{\abbr}[1]{\textsc{#1}}
\newcommand{\eg}{e.g.\,\xspace}
\newcommand{\ie}{i.e.\,\xspace}

% ===========================
% JUDGEMENT MARKERS (TEXT-MODE SAFE)
% ===========================
\newcommand{\judgesep}{\kern-0.15em}
\newcommand{\ungram}[1]{*\judgesep#1}
\newcommand{\marg}[1]{?\judgesep#1}
\newcommand{\odd}[1]{\#\judgesep#1}

% ===========================
% LINGUISTIC EXAMPLES
% ===========================
\usepackage{langsci-gb4e}
\makeatletter
\@ifundefined{noautomath}{}{\noautomath}
\makeatother

% ===========================
% BIBLIOGRAPHY
% ===========================
\usepackage[backend=biber,style=apa,natbib=true,doi=true,isbn=false,url=true]{biblatex}
\addbibresource{refs.bib}
\newcommand{\posscite}[1]{\citeauthor{#1}'s (\citeyear{#1})}
\usepackage{orcidlink}

% ===========================
% TITLE
% ===========================
\title{Grammaticality de-idealized:\\An etiological account of stable repertoire exclusion}
\author{Brett Reynolds \orcidlink{0000-0003-0073-7195}\\Humber Polytechnic \& University of Toronto\\\href{mailto:brett.reynolds@humber.ca}{brett.reynolds@humber.ca}}
\date{}

\begin{document}
\maketitle

\begin{abstract}
% TODO: Draft abstract once structure stabilizes.
% Core pitch: The constitutive account of grammaticality (what it IS) has been
% developed elsewhere. This companion paper asks the etiological question:
% why do C_t(u,c) trajectories settle where they do? Two complementary modules:
% (1) opportunity-sensitive preemption (niche-varying strength),
% (2) coordination equilibria (game-theoretic self-sustaining exclusion).
% Together they explain why some gaps are stable and others drift, without
% treating categorical constraints as primitives.
\end{abstract}

\noindent Keywords: grammaticality; repertoire exclusion; preemption; coordination equilibrium; conditioned stability; etiology

\section{Introduction}

% The constitutive question (what grammaticality IS) and the etiological
% question (why C_t trajectories settle where they do) are separable.
% The asterisk paper and the OVMG main paper handle the constitutive side.
% This paper develops the etiological module announced there.
%
% Key opening move: classic "categorical constraints" are not primitives
% but stable outcomes of ongoing dynamics. The question is what sustains
% them.

\section{The constitutive--etiological distinction}

% Brief recap of OVMG state theory:
%   G_t(u,c) = map(u,c) · K(u,c) · C_t(u,c)
% This paper focuses entirely on C_t: repertoire status.
% Why does C_t(u,c) ≈ 0 for some structurally viable, interpretable forms?
% That's the etiological question.

\section{Module 1: Opportunity-sensitive preemption}

% Preemption redescribed in OVMG terms.
% Not "does preemption exist?" but "what is its effective strength
% across niches, and how does it interact with processing difficulty
% and social evaluation?"
%
% Key formalization: preemption strength as a function of opportunity
% set robustness. High-frequency alternatives in robust opportunity
% sets sustain exclusion; niche-specific or low-frequency alternatives
% allow drift.
%
% Test case: classic "categorical" constraints as stable repertoire
% exclusion under strong preemption.

\subsection{Opportunity sets and preemption strength}

\subsection{Niche variation}

\subsection{Interaction with processing difficulty}

\section{Module 2: Coordination equilibria}

% Game-theoretic account (O'Connor 2019 and related).
% Communicative situations as payoff structures.
% Repertoire boundaries stabilize because they solve coordination
% problems.
%
% Why C_t ≈ 0 can persist for viable, interpretable forms:
% the coordination equilibrium excludes them. Unilateral deviation
% is penalized.
%
% The threshold τ(c) fits naturally: high-stakes = high coordination
% cost = sharper policing. Institutions encode equilibria as
% explicit gatekeeping.

\subsection{Communicative situations as payoff structures}

\subsection{Self-sustaining exclusion}

\subsection{The threshold as coordination cost}

\section{Integration: Why some gaps are stable and others drift}

% Bringing the two modules together.
% Preemption provides the proximate mechanism (what blocks the form).
% Coordination equilibria explain why the blocking persists
% (self-sustaining under deviation penalties).
%
% Predictions:
% - Gaps sustained by both modules are maximally stable
% - Gaps sustained by preemption alone drift when opportunity sets
%   shift (genre change, register mixing)
% - Gaps sustained by coordination alone weaken when coordination
%   payoffs diminish (informal contexts, in-group speech)

\section{Empirical predictions and disconfirmation conditions}

% Concrete, falsifiable predictions.
% What would have to be true for each module to fail?

\section{Conclusion}

\section*{Acknowledgements}

The large language models Claude Opus 4.5, ChatGPT 5.2 Pro, and Gemini 3 served as drafting and editing aids throughout the preparation of this paper. I am responsible for all theoretical claims, arguments, errors, and interpretive choices.

\newpage
\printbibliography

\end{document}
