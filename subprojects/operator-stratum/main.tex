% !TEX TS-program = xelatex
\documentclass[12pt,letterpaper]{article}

\usepackage{fontspec}
\setmainfont{EB Garamond}[
  Numbers=OldStyle,
  Ligatures=TeX,
]

\usepackage[
  letterpaper,
  margin=1in,
]{geometry}

\usepackage[british]{babel}
\usepackage[final]{microtype}

\usepackage{booktabs}
\usepackage{array}

\usepackage[style=american]{csquotes}

\usepackage{xcolor}
\definecolor{linkmaroon}{RGB}{128, 0, 32}

\usepackage{hyperref}
\hypersetup{
  colorlinks=true,
  linkcolor=linkmaroon,
  citecolor=linkmaroon,
  urlcolor=linkmaroon,
  pdftitle={The operator stratum},
  pdfauthor={Brett Reynolds},
}

\usepackage[backend=biber,style=apa,natbib=true,doi=true,isbn=false,url=true]{biblatex}
\addbibresource{refs.bib}

\usepackage{langsci-gb4e}
\makeatletter
\@ifundefined{noautomath}{}{\noautomath}
\makeatother

\newcommand{\term}[1]{\textsc{#1}}
\newcommand{\mention}[1]{\textit{#1}}

\title{The operator stratum: a value-based account of why clause structure behaves like tense and number}
\author{Brett Reynolds\\Humber Polytechnic \& University of Toronto}
\date{\today}

\begin{document}
\maketitle

\begin{abstract}
Grammaticality judgements show a striking selectivity: speakers routinely treat violations of clause structure, agreement, tense--aspect, polarity, or case as a distinct kind of wrongness, while treating accent shifts, many lexical substitutions, and register choices as matters of style, identity, or appropriateness rather than \enquote{grammar}. This paper proposes that this selectivity is not an artefact of English or of any particular syntactic theory. Rather, it reflects a cross-linguistically robust division in the \term{value} of linguistic resources. I argue that communities conventionalise a relatively small set of contrasts as \term{operators}: publicly accountable form--value relations that organise how utterances update shared commitments, allocate participant roles, and stabilise combinatorial possibility. These \term{operator} relations are what grammaticality talk tracks. Clause structure falls under this stratum because it is itself a system of operators (clause typing, argument linking, dependency management, and scope), not because it is autonomous from meaning. Accent and open-class lexical choice typically fall outside because they primarily contribute indexical and conceptual value that is negotiable, defeasible, and weakly tied to public uptake obligations, though phonological and gestural material can enter the operator stratum when grammaticalised (e.g.\ tone, interrogative prosody, sign-language nonmanuals). The proposal predicts where strong categorical judgements, repair behaviour, and rapid preemption dynamics should cluster across languages, and it reframes typological diversity as variation in which contrasts are recruited into the operator stratum.
\end{abstract}

\section{The puzzle: why \enquote{grammar} picks out some things and not others}

In ordinary metalinguistic practice, speakers distinguish at least three kinds of linguistic mismatch. Some mismatches are treated as straightforward errors of \enquote{grammar}, often with an accompanying sense that there is nothing to debate: wrong agreement, illicit clause structure, incompatible tense--aspect morphology, missing obligatory evidentials, or mis-selected question particles. Other mismatches are treated as infelicities of social positioning: an accent that indexes the wrong stance, a style that fails to fit the activity type, a form that sounds \enquote{too formal} or \enquote{too intimate}. Still other mismatches are treated as mere lexical choice: one could have used a more apt word, but the choice does not make the utterance ill-formed.

A familiar temptation is to reduce this to a contrast between \enquote{structure} and \enquote{use}: morphosyntax counts as grammar; accent and lexical choice do not. But that gloss is neither typologically adequate nor theoretically explanatory. Many languages grammaticalise prosody, phonation, or tone in ways that make them as obligatory and contrastive as agreement or tense \parencite{hyman2011,ladd2008}. Many grammatical categories are not universal: a language may lack tense marking \parencite{comrie1985} or number inflection \parencite{corbett2000} while still sustaining robust categorical judgements elsewhere. And some lexical items function as narrow operators, with distributional constraints and systematic scope effects indistinguishable from those of inflectional morphology \parencite{hoppertraugott2003,bybee2010}.

The core question is therefore not \enquote{why morphosyntax is special} in the sense of being autonomous or insulated. It is why certain form--value relations are treated as the community's basic meaning-making infrastructure, so that violations are policed as membership facts, while other relations are treated as negotiable resources for stance, identity, and content selection. The argument developed here is that this difference tracks a difference in \term{value} in the Saussurean sense: what a unit is, is its place in a system of contrasts \parencite{saussure1916}. Grammaticality judgements target a particular kind of value: \term{operator value}.

\section{Value, commitments, and why some contrasts are publicly accountable}

The starting point is a deliberately non-idealised view of grammar. A community's linguistic system is not merely a set of strings or derivations, but a repertoire of conventionalised form--value relations that enable coordination in interaction. The relevant sense of coordination is the one captured by work on common ground and grounding: interlocutors do not merely produce signals; they attempt to make commitments mutually recognisable and mutually ratifiable \parencite{stalnaker1978,stalnaker2002,clarkbrennan1991,clark1996}. This makes some aspects of linguistic form \term{publicly accountable} in a way that others are not.

Two observations matter.

First, interaction is organised around obligations of uptake and repair. When an utterance is not understood, or is understood under a suspect analysis, interlocutors systematically mobilise repair mechanisms \parencite{schegloffjeffersonsacks1977}. These mechanisms presuppose that some dimensions of form are designed to be recognisably \enquote{the same thing} across tokens and across speakers, because they are the dimensions on which the interactional work turns.

Second, public accountability is selective. Indexical meaning is real and systematic, but it is typically tolerated as variable, defeasible, and contextually negotiable. Accent and style are paradigmatic examples: they are central to social meaning, but they are not normally treated as licensing conditions on what an utterance can \enquote{count as} in the way that clause-typing, polarity, or evidential marking can be \parencite{eckert2008,eckert2012,agha2007,coupland2007}. Lexical choice is similarly accountable in an ethical or epistemic sense (one can be challenged for using a slur, or for choosing a misleading term), but this is a different sort of accountability than \enquote{this form is not in the repertoire}.

These observations invite a more explicit distinction. If \term{value} is always relational and contrastive, then the relevant question is which contrasts a community treats as foundational for public update in interaction. Those contrasts constitute what I will call the \term{operator stratum}.

\section{The operator stratum}

\subsection{Definition}

A form--value relation belongs to the \term{operator stratum} of a communicative situation if it satisfies both of the following conditions.

\textsc{(i) Public update condition.} The relation conventionally contributes to how an utterance updates shared commitments or allocates interactional roles (speaker, addressee, evidential access, epistemic authority, polarity, modality), in a way that is recognisably \emph{intended} rather than merely inferred. This includes not only propositional content, but clause-type and illocutionary organisation, which structure what counts as an appropriate next move \parencite{stalnaker2002,farkasbruce2010}.

\textsc{(ii) Repertoire condition.} The community treats the relation as part of its stable meaning-making resources: it is a conventional option with a restricted paradigm, and it is governed by distributional constraints whose violation is typically treated as a categorical mismatch rather than as an idiosyncratic stylistic choice.

The claim is not that operator resources are always morphological or always syntactic. It is that they are conventionalised as \emph{operators}: closed-system contrasts that function as control parameters for interpretation and interaction. Morphology and clause structure are privileged \emph{where they instantiate operators}, not by virtue of being a separate module.

\subsection{Diagnostics}

The operator stratum can be diagnosed empirically, and the diagnostics are explicitly cross-linguistic.

\textsc{Paradigmatic closure.} Operator relations typically occupy relatively small, enumerable paradigms: polarity contrasts, evidential sets, tense--aspect systems, agreement paradigms, switch-reference markers, question particles, complementiser inventories, and so on. Open-class lexical sets do not behave this way.

\textsc{Broad scope.} Operator choices often have clause-wide consequences, affecting argument linking, scope, clause type, or discourse update potential.

\textsc{High opportunity mass.} Many operator niches are encountered constantly in ordinary interaction: polar questioning, reference tracking, temporal anchoring, attribution of source. This produces strong preemption dynamics: a competitor that systematically wins in a large opportunity set quickly drives a rival option toward non-licensing \parencite{bybee2006,bybee2010}. The force of the diagnostic does not depend on any particular learning model; it depends on the fact that repeated structured non-occurrence is evidential.

\textsc{Repair sensitivity.} Operator mismatches are disproportionately likely to trigger repair sequences, often early and often with a \enquote{you can't say that} flavour rather than a negotiation of style or attitude \parencite{schegloffjeffersonsacks1977}. This predicts a difference in repair profiles across mismatch types, and it predicts cross-linguistic differences depending on which contrasts are grammaticalised.

\textsc{Neurocognitive separation in \emph{response profile}, not in module.} The proposal is compatible with domain-general accounts of prediction and error signalling, but it predicts that operator violations should systematically elicit distinct response profiles from conceptual anomalies, because operator mismatches interfere with the mapping from form to public update, rather than merely producing a surprising concept combination. This aligns with well-known ERP dissociations between semantic incongruity and morphosyntactic anomaly \parencite{kutashillyard1980,osterhoutholcomb1992,friederici2011} without implying that syntax is autonomous.

None of these diagnostics depends on English. If they are right, then where a language grammaticalises a contrast, that contrast should behave \enquote{like grammar} even when the exponent is tonal, prosodic, or gestural.

\section{Why clause structure belongs to the operator stratum}

Clause structure is often discussed as if it were a matter of arranging words. That is an English-biased picture. Cross-linguistically, clause structure is realised through diverse exponent types: affix order in polysynthetic systems, templatic morphology, clitic clusters, tone melodies, nonmanual marking, rigid word order, flexible order coupled with case, and various mixtures. The generalisation relevant here is not about linearisation. It is about the \term{operator value} of clausal architecture.

Clause structure packages at least four operator families.

\textsc{Clause type and response space.} Interrogatives, imperatives, declaratives, and related types encode different update potentials and different norms for what counts as a relevant next move. This is interactionally basic, and languages grammaticalise clause type using heterogeneous resources (particles, morphology, intonation, or combinations) \parencite{dryer2013,ladd2008}. A clause-typing mismatch therefore disrupts public coordination, not merely aesthetic preferences.

\textsc{Argument linking and participant roles.} Clause structure encodes which participant is presented as actor, undergoer, experiencer, and so on. The particular implementation varies typologically, but the operator function is stable: it constrains the space of recoverable role assignments, and it does so in a publicly accountable way.

\textsc{Dependency management and scope.} Relative clauses, complement structures, and long-distance dependencies are not simply compositional ornaments. They package how information is nested, attributed, and scoped. These are conditions on what an utterance publicly commits to and how it can be challenged or responded to \parencite{stalnaker2002,farkasbruce2010}.

\textsc{Reference tracking across clauses.} Many languages treat cross-clausal coherence as an operator domain. Switch-reference is a textbook case: a morphosyntactic system signals whether the subject of one clause is the same as or different from that of another, thereby reducing ambiguity in multi-clause sequences \parencite{vangijn2016}. The key point here is not the particular pivot (which varies) but the fact that a community can grammaticalise the tracking problem as operator value.

To illustrate the form of the claim, consider a canonical switch-reference pattern discussed in the typological literature, where a medial verb carries an identity vs.\ non-identity marker whose value is computed relative to a subsequent clause \parencite{vangijn2016}. The operator is not \enquote{a content word}. It is a clause-linking control parameter. Violations are therefore predicted to be treated as structural mismatches: the wrong marker makes the clause-linking update incoherent in the relevant conventions.

In this sense clause structure is operator-rich. It is a bundle of public control settings. That is why its violations feel like violations of the community's basic repertoire, in much the way that the wrong tense or agreement value does.

\section{Why tense and number behave similarly without being universal}

On the present account, tense and number behave \enquote{like grammar} in languages where they are part of the operator stratum, and they are irrelevant to grammaticality talk where they are not grammaticalised. This avoids a common slippage between \enquote{salience in familiar European languages} and \enquote{cross-linguistic necessity} \parencite{haspelmath2007}.

Tense marking is an especially transparent example. Where tense is grammaticalised, it functions as a conventional operator on temporal anchoring and discourse update, and violations are often treated as categorical mismatches \parencite{comrie1985}. Where tense is not grammaticalised, temporal anchoring is managed through other operator resources (aspect, evidential access, discourse particles) and through contextual inference. The proposal predicts that speakers in the latter case should not experience missing tense marking as \enquote{ungrammatical}, because there is no operator expectation to violate.

Number behaves analogously. In languages where number is obligatory in agreement or nominal morphology, it is part of the operator repertoire for tracking reference in public commitments, and its violation is a paradigmatic licensing failure. In languages where number is optional or limited, other resources fill the reference-tracking role, and number mismatches have different status \parencite{corbett2000}.

The point is not that operator systems are fixed across languages. It is that grammaticality judgements target operator systems \emph{wherever they are}.

\section{Why accent and lexical choice are usually different}

Accent and lexical choice are not \enquote{less meaningful}. They typically have different kinds of value.

\subsection{Indexical value and negotiability}

Accent, style, and register contribute indexical value: they position speakers relative to social categories, stances, and activity types \parencite{silverstein1976,eckert2008,agha2007,coupland2007}. Indexical value is often rich and structured, but it is not typically implemented as a small closed paradigm with obligatory selection conditions tied to public update. It is negotiable in ways operators are not. Interlocutors can reinterpret an accent shift as play, accommodation, quotation, or stance work without treating it as a licensing failure. The accountability regime is therefore different: style can be challenged, but it is not ordinarily policed as \enquote{not in the repertoire}.

This is not a claim that indexicality is optional in any substantive sense. It is a claim about \emph{how communities conventionalise it}. Much indexical meaning is \emph{meta-pragmatic}: it becomes salient through social ideologies and enregisterment processes \parencite{agha2007,eckert2012}. That makes it both powerful and variable, but it rarely functions as a control parameter that must take a specific value for the utterance to count as a recognised move.

\subsection{Conceptual value and open classes}

Open-class lexical items primarily contribute conceptual content, and the system is designed to be expansible and to tolerate innovation. Lexical choice can be infelicitous, misleading, or socially harmful, but it is seldom treated as a failure of combinatorial licensing. Indeed, lexical innovation is one of the routine mechanisms by which communities extend their repertoire, and it typically proceeds without the categorical judgement profile associated with operator violations.

This is why \enquote{wrong word} and \enquote{wrong clause structure} often feel qualitatively different even when both are understood. A wrong operator setting threatens the public recognisability of the intended update; a non-optimal lexical choice typically does not.

\subsection{The crucial qualification: phonology and lexicon can become operators}

Nothing in this account entails a substance-based boundary. Phonological material can be an operator exponent, and lexical items can be operators when they grammaticalise.

Tonal exponents that mark tense, focus, or polarity are paradigmatic operator systems, and they should attract categorical judgements accordingly \parencite{hyman2011}. Intonational contours that conventionalise clause type are operator exponents, and they should pattern like question particles, not like accent \parencite{ladd2008}. In signed languages, nonmanual markers that encode clause type or operator scope similarly belong in the operator stratum \parencite{sandlerlillomartin2006}.

Conversely, lexical items can undergo grammaticalisation and become operator-like: they shrink paradigmatically, lose descriptive content, and acquire scope and distributional constraints \parencite{hoppertraugott2003,bybee2010}. On this view, the difference between lexical and operator value is a difference in \emph{conventional role} rather than in form.

This qualification is not a complication; it is the typological payoff. It predicts that the boundary tracked by grammaticality talk is not \enquote{morphosyntax vs.\ phonology} or \enquote{syntax vs.\ lexicon}. It is \enquote{operator vs.\ non-operator}.

\section{Typological consequences: what varies is which contrasts are recruited as operators}

The operator stratum is a comparative concept: it identifies a functional role that can be realised by different structures in different languages \parencite{haspelmath2007}. This frames typological diversity in a way that avoids smuggling in English categories as universals.

Two domains are particularly instructive.

\subsection{Evidentiality and epistemic authority}

In many languages, evidentiality is grammaticalised and obligatory in a way that makes it an operator on epistemic access and public commitment \parencite{aikhenvald2004}. Where this is the case, evidential mismatches should pattern as operator violations: they are not merely \enquote{odd} or \enquote{misleading}; they are failures to supply a required value in a closed system. This predicts strong categoricality in judgement and robust repair sensitivity. Where evidentiality is not grammaticalised, similar meanings are available through lexical or periphrastic means, but their status is different: they can be challenged as deceptive or inappropriate without being treated as structurally illicit.

\subsection{Egophoricity and perspective management}

Tibetic languages and others have been argued to grammaticalise distinctions tied to perspective, authority, or ego involvement, with ongoing debates about how to characterise these systems \parencite{tournadre2008,floydnorcliffesanroque2018}. Regardless of the preferred analysis, these systems are prime candidates for operator status: they govern whose epistemic position is encoded as the relevant anchor for a clause's update potential. If so, they should behave like grammar in the same sense as agreement or polarity does: restricted paradigms, distributional constraints, and a categorical judgement profile when violated in the relevant communicative situations.

The operator-stratum view predicts a convergence between such perspective systems and more familiar operator categories. It also predicts that where languages do \emph{not} recruit perspective into the operator stratum, similar meanings will remain expressible but will be policed differently (as pragmatic misalignment rather than licensing failure).

\section{Grammaticality judgements as a social technology}

If grammaticality talk targets operator value, then grammaticality judgements are themselves a social technology: a way of policing the operator repertoire of a communicative situation. This aligns with two independently motivated ideas.

First, grammatical conventions are community property: they are the stabilised solutions that allow rapid coordination under uncertainty \parencite{clarkbrennan1991,stalnaker2002}. Second, the repair system treats some departures as requiring correction and others as negotiable \parencite{schegloffjeffersonsacks1977}. Operator violations are predicted to sit at the intersection: they are departures that threaten coordination precisely because they disrupt publicly accountable control settings.

This framing also clarifies why categoricality is common but not inevitable. Some operator systems are heavily conventionalised, high-opportunity, and tightly constrained, which makes non-licensing stable and judgements sharp. Others are lower-opportunity or contested across overlapping norm centres, which makes judgements more variable. The key claim is that \emph{wherever categoricality emerges}, it should cluster around operator value.

\section{Predictions and research strategies}

The proposal generates tractable empirical predictions across subdisciplines, which is part of its point: it is meant to be usable by typologists, sociolinguists, psycholinguists, and interaction analysts.

\textsc{(1) Repair asymmetries.} In conversational corpora, operator mismatches should show higher rates of other-initiated repair and more direct formulation as \enquote{can't say} or \enquote{that's not how it goes} than indexical mismatches, which should more often be negotiated as stance or appropriateness \parencite{schegloffjeffersonsacks1977}. Cross-linguistically, the targets of repair should correlate with what is grammaticalised.

\textsc{(2) Satiation asymmetries.} Repetition should attenuate anomaly responses more readily for non-operator mismatches and for low-entrenchment operator candidates than for high-opportunity, strongly preempted operator gaps. This follows from the role of opportunity mass in stabilising non-licensing \parencite{bybee2006,bybee2010}.

\textsc{(3) Phonological operator effects.} In languages where tone or intonation functions as an operator exponent, violations should elicit judgement and processing profiles more similar to morphosyntactic anomalies than to accent shifts, consistent with the idea that the relevant factor is operator value, not segmental content \parencite{hyman2011,ladd2008}.

\textsc{(4) Sociolinguistic stratification without grammaticalisation.} Accent and register differences should show systematic indexical fields and enregisterment dynamics \parencite{eckert2008,agha2007}, but the metalinguistic categorisation as \enquote{grammar} should be predicted to arise mainly when the indexical resource becomes part of a closed operator system (e.g.\ a grammaticalised honorific agreement paradigm, not merely polite lexical choice).

\textsc{(5) Processing signatures as responses to operator failure.} ERP and related measures should distinguish operator mismatches from conceptual anomalies in ways that are consistent with their different roles in public update, without requiring a modular syntax/semantics division \parencite{kutashillyard1980,osterhoutholcomb1992,friederici2011}.

None of these predictions presupposes a particular formal architecture. They presuppose only that communities conventionalise a subset of contrasts as operator settings for public coordination.

\section{Conclusion}

The motivating puzzle was why clause structure is treated as \enquote{grammar} in much the same way as tense or number marking is, while accent and much lexical choice are treated differently. The proposed answer is that grammaticality talk tracks a difference in value: the operator stratum. Clause structure belongs to it because it is itself a system of operators governing public update, role allocation, scope, and cross-clausal coherence. Tense and number belong where they are grammaticalised as operators. Accent and open-class lexical choice typically do not, because they primarily contribute indexical and conceptual value whose accountability regime is different, though phonological and lexical material can enter the operator stratum when grammaticalised.

This reframing makes three payoffs explicit. It de-Englishes the problem by predicting that \enquote{what counts as grammar} varies with which contrasts are grammaticalised, while preserving the generalisation that grammaticality targets operator value. It makes room for phonological and gestural operator exponents without weakening the empirical distinction between operator mismatch and style mismatch. And it yields a research programme with converging evidence streams: typological distribution, interactional repair, and processing signatures.

If grammaticality is the community's name for stability in its operator repertoire, then the special status of clause structure is not evidence for autonomous syntax. It is evidence for the social fact that some form--value relations function as the infrastructure of shared meaning-making.

\printbibliography
\end{document}
