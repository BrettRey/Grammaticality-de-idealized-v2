% !TEX TS-program = xelatex
\documentclass[12pt,letterpaper]{article}

\input{../../../../.house-style/preamble.tex}

% Add pdftitle
\hypersetup{pdftitle={The operator stratum: a contrastive-value account of why clause structure is treated like tense and number}}

\title{The operator stratum: a contrastive-value account of why clause structure is treated like tense and number}
\author{Brett Reynolds\\Humber Polytechnic \& University of Toronto}
\date{\today}

\begin{document}
\maketitle

\begin{abstract}
Grammaticality judgments show a striking selectivity: speakers routinely treat violations of clause structure, agreement, tense--aspect, polarity, or case as a distinct kind of wrongness, while treating accent shifts, many lexical substitutions, and register choices as matters of style, identity, or appropriateness rather than \enquote{grammar}. This paper proposes that this selectivity reflects a cross-linguistically robust division in linguistic infrastructure. Communities conventionalize three layers of resources: expression-shape constraints that make an utterance recognisable as a token of the system (phonotactics, morphotactics); \term{operators}~-- closed-paradigm contrasts that configure how utterances update shared commitments, allocate participant roles, and constrain uptake; and payload resources (open-class lexicon, indexical field) that are negotiable and extensible. Grammaticality talk targets operators in its \enquote{you can't say that (as a move)} sense; separate but related categorical reactions (\enquote{not a word}) target expression-shape constraints. From an information-theoretic perspective, operators function as protocol headers rather than payload content: they carry few bits in themselves but cause large entropy reduction downstream, which explains why a wrong value can have disproportionate coordination costs. Clause structure belongs to the operator stratum because it packages operators (clause typing, argument linking, dependency management, scope), not because it's autonomous from meaning. Accent and open-class lexical choice typically fall outside because they primarily contribute indexical and conceptual value whose accountability regime is different~-- though phonological and gestural material can enter the operator stratum when grammaticalized. The proposal predicts where categorical judgments, distinct processing signatures, and preemption dynamics should cluster across languages, and it reframes typological diversity as variation in which contrasts are recruited as operators.
\end{abstract}

\section{The puzzle: why \enquote{grammar} picks out some things and not others}

In ordinary metalinguistic practice, speakers distinguish at least three kinds of linguistic mismatch. Some mismatches are treated as straightforward errors of \enquote{grammar}, often with an accompanying sense that there's nothing to debate: wrong agreement, illicit clause structure, incompatible tense--aspect morphology, missing obligatory evidentials, or mis-selected question particles. Other mismatches are treated as infelicities of social positioning: an accent that indexes the wrong stance, a style that fails to fit the activity type, a form that sounds \enquote{too formal} or \enquote{too intimate}. Still other mismatches are treated as mere lexical choice: one could have used a more apt word, but the choice doesn't make the utterance ill-formed. (Speakers do label accent and dialect variation as \enquote{bad grammar} for social reasons, but the interactional crash profile differs~-- a point taken up below.)

A familiar temptation is to reduce this to a contrast between \enquote{structure} and \enquote{use}: morphosyntax counts as grammar; accent and lexical choice don't.\footnote{This is the moment at which introductory diagrams draw a reassuring boundary; languages then spend the rest of the semester crossing it.} But that gloss is neither typologically adequate nor theoretically explanatory. Many languages grammaticalize prosody, phonation, or tone in ways that make them as obligatory and contrastive as agreement or tense \citep{hyman2011,ladd2008}. Many grammatical categories aren't universal: a language may lack tense marking \citep{comrie1985} or number inflection \citep{corbett2000} while still sustaining robust categorical judgments elsewhere. And some lexical items function as narrow operators, with distributional constraints and systematic scope effects indistinguishable from those of inflectional morphology \citep{hoppertraugott2003,bybee2010}.

The core question is not why morphosyntax is insulated from meaning, but why certain meanings are installed as morphosyntax~-- why some form--value relations become the community's non-negotiable infrastructure, policed as membership facts, while others remain negotiable resources for stance, identity, and content selection. The argument developed here is that this difference tracks a difference in \term{value}. The term \term{value} is used here in the Saussurean sense of \mention{valeur} \citep{saussure1916}: the identity of a linguistic unit is constituted not by intrinsic properties but by its position in a system of contrasts~-- what it patterns with, what it opposes, what interpretations it makes available. It isn't to be confused with feature-value assignment or decision-theoretic value. Throughout, I treat grammatical knowledge as a conditioned \textbf{form--value relation}; when this relation is sufficiently stable in a communicative situation~-- i.e.\ when a dominant value is reliably recoverable and socially licensed~-- I refer to it informally as an established \textbf{form--value relationship}. Grammaticality judgments target a particular kind of value: \term{operator value}.

\section{Value, commitments, and why some contrasts are publicly accountable}

The starting point is a deliberately non-idealized view of grammar. A community's linguistic system isn't merely a set of strings or derivations, but a repertoire of conventionalized form--value relations that enable coordination in interaction. The relevant sense of coordination is the one captured by work on common ground and grounding: interlocutors don't merely produce signals; they attempt to make commitments mutually recognizable and mutually ratifiable \citep{stalnaker1978,stalnaker2002,clarkbrennan1991,clark1996}. This makes some aspects of linguistic form \term{publicly accountable} in a way that others aren't.

Two observations matter.

First, interaction is organized around obligations of uptake and repair. When an utterance isn't understood, or is understood under a suspect analysis, interlocutors systematically mobilize repair mechanisms \citep{schegloffjeffersonsacks1977}. These mechanisms presuppose that some dimensions of form are designed to be recognizably \enquote{the same thing} across tokens and across speakers, because they're the dimensions on which the interactional work turns.

Second, public accountability is selective. Indexical meaning is real and systematic, but it's typically tolerated as variable, defeasible, and contextually negotiable. Accent and style are paradigmatic examples: they're central to social meaning, but they aren't normally treated as licensing conditions on what an utterance can \enquote{count as} in the way that clause-typing, polarity, or evidential marking can be \citep{eckert2008,eckert2012,agha2007,coupland2007}. Lexical choice is similarly accountable in an ethical or epistemic sense (one can be challenged for using a slur, or for choosing a misleading term), but this is a different sort of accountability than \enquote{this form isn't in the repertoire}.

These observations invite a more explicit distinction. If \term{value} is always relational and contrastive, then the relevant question is which contrasts a community treats as foundational for public update in interaction. Those contrasts constitute what I will call the \term{operator stratum}.

\section{The operator stratum}

\subsection{Definition}

A form--value relation belongs to the \term{operator stratum} of a communicative situation if it satisfies both of the following conditions.

\textsc{(i) Public update condition.} The relation conventionally contributes to how an utterance updates shared commitments or allocates interactional roles, but in a specific way: it functions as an \emph{instruction} for the update~-- an instruction that constrains what counts as an appropriate next move, and a next move that in turn constrains what can follow it \citep{stalnaker2002,farkasbruce2010}. Clause-type, illocutionary force, and evidential anchoring are paradigm cases. It's distinguished from open-class meaning by being conventionally encoded as an update instruction~-- a signal that constrains uptake, ratification, and the space of licit next moves~-- rather than merely contributing descriptive content whose interactional consequences are left to contextual inference.

\textsc{(ii) Repertoire condition.} The community treats the relation as part of its stable meaning-making resources: it's a conventional option with a restricted paradigm, and it's governed by distributional constraints whose violation is typically treated as a categorical mismatch. This excludes idioms and fixed collocations, which may be rigid but don't function as combinatorial values in a contrastive system.

The claim isn't that operator resources are always morphological or always syntactic. It's that they're conventionalized as \emph{operators}: closed-system contrasts that function as control parameters for interpretation and interaction. Morphology and clause structure are privileged \emph{where they instantiate operators}, not by virtue of being a separate module.

Operators are intensely contrastive, but at a different functional level than phonemes. Phonemic value is primarily discriminative: its semiotic job is to keep signs apart. Operator value is contrastive at the level of interactional control: it selects among publicly accountable update settings (clause type, polarity, evidential anchoring, and so on). Both are paradigmatic, but they serve different functions in the system.

A useful analogy: operators function as protocol headers rather than payload content. They configure how an utterance is to be processed~-- what kind of move it is, who is committed to what, how subsequent contributions are constrained~-- rather than contributing the conceptual substance of what's being communicated. This framing helps distinguish three layers of linguistic infrastructure: expression-shape constraints that make an utterance recognisable as a token of the system (phonotactics, morphotactics, morphophonological allomorphy, prosodic well-formedness); operator constraints that configure public update; and payload resources (open-class lexicon, indexical field) that are negotiable and extensible. The operator stratum is the middle layer. Phonotactic failures crash recognisability: \emph{that's not a word}. Operator failures crash coordination: \emph{you can't say that}. Payload failures invite negotiation: \emph{did you mean something else?} Operators must be mutually known in the way a radio frequency must be shared: if sender and receiver aren't tuned to the same channel, content doesn't get through, no matter how good it is.

\begin{table}[ht]
\centering
\small
\begin{tabular}{@{}llll@{}}
\toprule
\textbf{Layer} & \textbf{Function} & \textbf{Mismatch consequence} & \textbf{Example} \\
\midrule
Expression-shape & Token recognisable as system member & \enquote{Not a word} & phonotactic violation \\
Operator & Configures public update & \enquote{Can't say that (as a move)} & clause-type error \\
Payload & Contributes content/stance & Negotiation, not rejection & odd lexical choice \\
\bottomrule
\end{tabular}
\caption{Three layers of linguistic infrastructure and their characteristic mismatch profiles.}
\label{tab:layers}
\end{table}

The three-layer distinction is broader than this paper's central puzzle, which concerns the operator layer specifically: why clause structure clusters with tense, number, and agreement as targets of categorical grammaticality talk, rather than with accent or lexical choice. The answer developed below is that clause structure, like tense and agreement, packages operator contrasts~-- publicly accountable control settings for update, role allocation, and uptake~-- and that's what makes it behave \enquote{like grammar}.

\subsection{Diagnostics}

The operator stratum can be diagnosed empirically, and the diagnostics are explicitly cross-linguistic.

\textsc{Paradigmatic closure.} Operator relations typically occupy relatively small, enumerable paradigms: polarity contrasts, evidential sets, tense--aspect systems, agreement paradigms, switch-reference markers, question particles, complementiser inventories, and so on. Open-class lexical sets don't behave this way.

\textsc{Broad scope.} Operator choices often have clause-wide consequences, affecting argument linking, scope, clause type, or discourse update potential. In public-update terms: they constrain which next moves are relevant (clause type), who is committed to what (evidential anchoring, argument structure), and how the contribution can be challenged or elaborated (dependency structure, polarity).

\textsc{Opportunity mass and preemption.} Many operator niches are encountered constantly in ordinary interaction, which produces strong preemption dynamics: a competitor that systematically wins in a large opportunity set quickly drives a rival option toward non-licensing \citep{bybee2006,bybee2010}. Where opportunity mass is lower, the same kinds of relations may still be operator-like, but categoricality and preemption effects are predicted to be weaker and more register-bound.

\textsc{Neurocognitive separation in \emph{response profile}, not in module.} The proposal is compatible with domain-general accounts of prediction and error signalling, but it predicts that operator violations should systematically elicit distinct response profiles from conceptual anomalies, because operator mismatches interfere with the mapping from form to public update, rather than merely producing a surprising concept combination. This aligns with well-known ERP dissociations between semantic incongruity and morphosyntactic anomaly \citep{kutashillyard1980,osterhoutholcomb1992,friederici2011} without implying that syntax is autonomous.

These diagnostics characterise operator systems independently of the behavioural phenomena to be explained; categorical judgments and repair profiles are predicted consequences, not defining criteria.

None of these diagnostics depends on English. If they're right, then where a language grammaticalizes a contrast, that contrast should behave \enquote{like grammar} even when the exponent is tonal, prosodic, or gestural.

\section{Why clause structure belongs to the operator stratum}

Clause structure is often discussed as if it were a matter of arranging words. That's an English-biased picture. Cross-linguistically, clause structure is realized through diverse exponent types: affix order in polysynthetic systems, templatic morphology, clitic clusters, tone melodies, nonmanual marking, rigid word order, flexible order coupled with case, and various mixtures. The generalization relevant here isn't about linearization. It's about the \term{operator value} of clausal architecture.

Clause structure packages at least four operator families.

\textsc{Clause type and response space.} Interrogatives, imperatives, declaratives, and related types encode different update potentials and different norms for what counts as a relevant next move. This is interactionally basic, and languages grammaticalize clause type using heterogeneous resources (particles, morphology, intonation, or combinations) \citep{dryer2013,ladd2008}. A clause-typing mismatch disrupts public coordination, not merely aesthetic preferences.

\textsc{Argument linking and participant roles.} Clause structure encodes which participant is presented as actor, undergoer, experiencer, and so on. The particular implementation varies typologically, but the operator function is stable: it constrains the space of recoverable role assignments, and it does so in a publicly accountable way.

\textsc{Dependency management and scope.} Relative clauses, complement structures, and long-distance dependencies aren't simply compositional ornaments. They package how information is nested, attributed, and scoped. These are conditions on what an utterance publicly commits to and how it can be challenged or responded to \citep{stalnaker2002,farkasbruce2010}.

\textsc{Reference tracking across clauses.} Many languages treat cross-clausal coherence as an operator domain. Switch-reference is a textbook case: a morphosyntactic system signals whether the subject of one clause is the same as or different from that of another, thereby reducing ambiguity in multi-clause sequences \citep{vangijn2016}. The key point here isn't the particular pivot (which varies) but the fact that a community can grammaticalize the tracking problem as operator value.

To illustrate the form of the claim, consider a canonical switch-reference pattern discussed in the typological literature, where a medial verb carries an identity vs.\ non-identity marker whose value is computed relative to a subsequent clause \citep{vangijn2016}. The operator isn't \enquote{a content word}. It's a clause-linking control parameter. Violations are predicted to be treated as structural mismatches: the wrong marker makes the clause-linking update incoherent in the relevant conventions.

A second illustration comes from Japanese sentence-final particles. In many registers, the particle \mention{ka} conventionally marks interrogative force, while declaratives are unmarked or carry other particles; intonation interacts with these choices, and colloquial speech permits additional variation. The particle doesn't contribute conceptual content so much as it configures what kind of move the utterance counts as and what responses are licensed. In contexts where \mention{ka} is the expected interrogative exponent and intonation doesn't independently supply an interrogative reading, omitting it yields a different illocutionary construal rather than a merely stylistic variant: the addressee's uptake obligations shift. This is the sense in which clause-typing is an operator: a small, conventionalised contrast with categorical consequences for public coordination.

In this sense clause structure is operator-rich. It's a bundle of public control settings. That's why its violations feel like violations of the community's basic repertoire, in much the way that the wrong tense or agreement value does.

\section{Why tense and number behave similarly without being universal}

On the present account, tense and number behave \enquote{like grammar} in languages where they're part of the operator stratum, and they're irrelevant to grammaticality talk where they aren't grammaticalized. This avoids a common slippage between \enquote{salience in familiar European languages} and \enquote{cross-linguistic necessity} \citep{haspelmath2007}.

Tense marking is an especially transparent example. Where tense is grammaticalized, it functions as a conventional operator on temporal anchoring and discourse update, and violations are often treated as categorical mismatches \citep{comrie1985}. Where tense isn't grammaticalized, temporal anchoring is managed through other operator resources (aspect, evidential access, discourse particles) and through contextual inference. The proposal predicts that speakers in the latter case shouldn't experience missing tense marking as \enquote{ungrammatical}, because there's no operator expectation to violate.

Number behaves analogously. In languages where number is obligatory in agreement or nominal morphology, it's part of the operator repertoire for tracking reference in public commitments, and its violation is a paradigmatic licensing failure. In languages where number is optional or limited, other resources fill the reference-tracking role, and number mismatches have different status \citep{corbett2000}.

The point isn't that operator systems are fixed across languages. It's that grammaticality judgments target operator systems \emph{wherever they are}.

\section{Why accent and lexical choice are usually different}

Accent and lexical choice aren't \enquote{less meaningful}. They typically have different kinds of value.

\subsection{Indexical value and negotiability}

Accent, style, and register contribute indexical value: they position speakers relative to social categories, stances, and activity types \citep{silverstein1976,eckert2008,agha2007,coupland2007}. Indexical value is often rich and structured, but it isn't typically implemented as a small closed paradigm with obligatory selection conditions tied to public update. It's negotiable in ways operators aren't. Interlocutors can reinterpret an accent shift as play, accommodation, quotation, or stance work without treating it as a licensing failure. The accountability regime is different: style can be challenged, but it isn't ordinarily policed as \enquote{not in the repertoire}.

This isn't a claim that indexicality is optional in any substantive sense. It's a claim about \emph{how communities conventionalize it}. Much indexical meaning is \emph{meta-pragmatic}: it becomes salient through social ideologies and enregisterment processes \citep{agha2007,eckert2012}. That makes it both powerful and variable, but it rarely functions as a control parameter that has to take a specific value for the utterance to count as a recognized move.

\subsection{Conceptual value and open classes}

Open-class lexical items primarily contribute conceptual content, and the system is designed to be expansible and to tolerate innovation. Lexical choice can be infelicitous, misleading, or socially harmful, but it's seldom treated as a failure of combinatorial licensing. Lexical innovation is one of the routine mechanisms by which communities extend their repertoire, and it typically proceeds without the categorical judgment profile associated with operator violations.

\enquote{Open-endedness} has two dimensions that matter here. Lexical inventories are open in the sense of expandable type-sets: innovation is normal and the system tolerates new entries. Compositional semantics is open-ended in a different sense: unbounded productivity from combining a finite stock. Operators lack the first kind of openness (paradigmatic closure is one of the diagnostics), but they participate fully in the second: operator meanings compose~-- negation scopes, evidentials anchor commitments, clause type constrains response space. Because the inventory is closed and the contribution is control-like, the community can stabilize sharp licensing expectations around them.

This is why \enquote{wrong word} and \enquote{wrong clause structure} often feel qualitatively different even when both are understood. A wrong operator setting threatens the public recognizability of the intended update; a non-optimal lexical choice typically doesn't.

\subsection{The crucial qualification: phonology and lexicon can become operators}

Nothing in this account entails a substance-based boundary. Phonological material can be an operator exponent, and lexical items can be operators when they grammaticalize.

Tonal exponents that mark tense, focus, or polarity are paradigmatic operator systems, and they should attract categorical judgments accordingly \citep{hyman2011}. Intonational contours that conventionalize clause type are operator exponents, and they should pattern like question particles, not like accent \citep{ladd2008}. In signed languages, nonmanual markers that encode clause type or operator scope similarly belong in the operator stratum \citep{sandlerlillomartin2006}.

Conversely, lexical items can undergo grammaticalization and become operator-like: they shrink paradigmatically, lose descriptive content, and acquire scope and distributional constraints \citep{hoppertraugott2003,bybee2010}. Many \enquote{lexical} items already behave as operators without undergoing grammaticalization in any diachronic sense: function words, particles, auxiliaries, complementizers, and polarity items occupy restricted paradigms, show distributional licensing, take broad scope, and attract repair sensitivity. They're lexical in form but operator-like in function. On this view, the difference between lexical and operator value is a difference in \emph{conventional role} rather than in substance.

This qualification is the typological payoff. The boundary isn't morphosyntax vs.\ phonology; it isn't syntax vs.\ lexicon. It's operator vs.\ non-operator~-- and operator is defined by function, not by form.

Phonotactics sharpens this point. Phonotactic constraints are repertoire-like: they're conventionalized, relatively stable, and violations are often treated categorically (\enquote{not a possible word}). But they don't satisfy the public-update condition. What they regulate is whether a token can be recognized as a well-formed expression-shape, not whether it specifies an update instruction correctly. The metalinguistic reaction profile is correspondingly different: operator failures invite \enquote{you can't say that (as a move)}; phonotactic failures invite \enquote{that's not a word / what did you say?} This suggests a three-way distinction rather than a two-way: operator violations crash publicly accountable control settings; phonotactic violations crash recognizability as a token of the system; lexical and pragmatic infelicities preserve recognizability but invite negotiation about content, stance, or appropriateness.

\subsection{An information-theoretic frame}

Why should closed, low-bit paradigms attract categorical policing while open-class resources remain negotiable? An information-theoretic perspective sharpens the answer by separating three properties that linguistic coding systems trade off: the size of the contrast set, the variability tolerated within a type, and how much context reduces uncertainty about the next element.

Phones and phonemes have a small contrast set but high within-category variability~-- the classic \enquote{lack of invariance} problem in speech perception. The system works because categories are recovered from multiple cues spread over time and conditioned on context, not because each phoneme has a stable acoustic signature \citep{libermanetal1967}. Redundancy and phonotactic predictability do much of the work.

Lexemes occupy a huge contrast set~-- tens of thousands of types~-- but the system manages this by distributing information across sequences (many segments per word) and across context (high predictability in ordinary use). Word length tracks average information content better than raw frequency \citep{piantadositilygibson2011}, consistent with a code that allocates more signal to less predictable items.

Operators occupy a small contrast set with constrained exponents. They carry few bits in themselves, but they cause large entropy reduction downstream: clause type constrains the response space, polarity flips entailment relations, case and agreement constrain role assignment. They function as control settings~-- protocol headers~-- rather than payload content. A wrong value can have disproportionate coordination costs even if it's low-information in the Shannon sense.

This explains a puzzle: grammatical marking is often locally redundant.\footnote{Engineers call this redundancy; linguists call it agreement.} \textcite{mahowaldetal2023} show that across 30 languages, subjecthood in simple transitive clauses can usually be inferred from lexical content alone. Overt grammatical cues are redundant in a majority of everyday cases~-- but crucial for the minority where lexical semantics doesn't settle the roles (reversible events, non-prototypical assignments). Operators are maintained even when locally predictable because redundancy buys robustness and supports the communication of rare or reversible meanings.

The same logic explains why grammatical material can be phonetically reduced. High-frequency, high-predictability items~-- often function words and inflections~-- undergo reduction precisely because context carries the load \citep{jaeger2010}. Operators can tolerate variable realisation without losing their status as operators, because what matters is their role in a small, closed paradigm, not the phonetic stability of their exponents.

This information-theoretic picture connects directly to the opportunity mass diagnostic introduced earlier. High opportunity mass provides dense data for estimating distributions, which stabilises preemption dynamics and sharpens licensing boundaries \citep{bybee2006,bybee2010}. The result is categorical policing: once the community converges on a small, high-frequency paradigm with large downstream entropy effects, deviations are readily detected and reliably sanctioned.

\section{Typological consequences: what varies is which contrasts are recruited as operators}

The operator stratum is a comparative concept: it identifies a functional role that can be realized by different structures in different languages \citep{haspelmath2007}. This frames typological diversity in a way that avoids smuggling in English categories as universals.

Four domains are particularly instructive.

\subsection{Instructive domains of variation}

\textsc{Evidentiality and epistemic authority.} In many languages, evidentiality is grammaticalized and obligatory in a way that makes it an operator on epistemic access and public commitment \citep{aikhenvald2004}. Where this is the case, evidential mismatches should pattern as operator violations: they're failures to supply a required value in a closed system, not merely \enquote{odd} or \enquote{misleading}. This predicts strong categoricality in judgment and robust repair sensitivity. Where evidentiality isn't grammaticalized, similar meanings are available through lexical or periphrastic means, but their status is different: they can be challenged as deceptive or inappropriate without being treated as structurally illicit.

\textsc{Egophoricity and perspective management.} Tibetic languages and others have been argued to grammaticalize distinctions tied to perspective, authority, or ego involvement, with ongoing debates about how to characterize these systems \citep{tournadre2008,floydnorcliffesanroque2018}. Regardless of the preferred analysis, these systems are prime candidates for operator status: they govern whose epistemic position is encoded as the relevant anchor for a clause's update potential. If so, they should behave like grammar in the same sense as agreement or polarity does: restricted paradigms, distributional constraints, and a categorical judgment profile when violated in the relevant communicative situations.

The operator-stratum view predicts a convergence between such perspective systems and more familiar operator categories. It also predicts that where languages do \emph{not} recruit perspective into the operator stratum, similar meanings will remain expressible but will be policed differently (as pragmatic misalignment rather than licensing failure).

\textsc{Phonological operators vs.\ accent.} The operator/indexical distinction cuts across the substance of the signal. Tonal languages provide a crucial test case. In many Bantu languages, specific tonal melodies mark tense, aspect, or polarity \citep{hyman2011}. These are operators, not \enquote{accent} features. Schematically, a language may distinguish tense values primarily through tonal melodies on the verb~-- one pattern for remote past, another for recent past~-- with minimal segmental differentiation between the forms.

A speaker who produces the wrong tone in these contexts need not be heard as merely \enquote{accented}; they may instead be heard as having selected the wrong tense value or polarity setting, inviting repair of the \enquote{did you mean X or Y?} type. This contrasts with sociolinguistic variation in the same languages, where tonal realization may shift to index region or age without affecting the operator value. The public update criterion distinguishes the two: the operator tone is an instruction for temporal/logical anchoring; the sociolinguistic variant is a performance of identity.

\textsc{Clause-linking: switch-reference as instruction.} Switch-reference systems exemplify how clause structure functions as an operator mechanism beyond simple word order. In languages with these systems, a medial verb is obligatorily marked to indicate whether its subject is coreferential with the subject of the following clause \citep{vangijn2016}.

\begin{exe}
\ex
\begin{xlist}
\ex \text{Verb-SS} \dots \text{Verb} (Instruction: \textit{Keep same participant active})
\ex \text{Verb-DS} \dots \text{Verb} (Instruction: \textit{Switch participant role})
\end{xlist}
\end{exe}

This marker is an explicit instruction for reference tracking: it constrains the assignment of participant roles in the upcoming increment. A mismatch~-- using the same-subject marker when the subjects are different~-- disrupts reference assignment and clause linkage; if repair occurs at all, it targets tracking failure rather than stance or stylistic appropriateness. This confirms that \enquote{public update} involves specific structural control settings, not just general meaningfulness.

\subsection{The boundary case: T/V and grammaticalized social deixis}

T/V distinctions (honorific pronouns and agreement) present the sharpest boundary case. They encode social positioning (indexical value), but in many languages they're paradigmatically closed and morphologically obligatory. Under the present account, this is exactly what's expected when indexical features enter the operator stratum. When a distinction like \enquote{addressee status} is grammaticalized as part of the agreement paradigm (e.g., distinct 2nd-person plural forms used for singular polite address), it acquires operator status. It becomes a control setting for interactional role allocation. Violations then cease to be merely \enquote{rude} (a social infraction) and become \enquote{ungrammatical} (a licensing failure), eliciting the categorical judgment profile. This illustrates the channel: indexical value becomes operator value when it's recruited into the closed-set infrastructure of public update.

\section{Grammaticality judgments as a social technology}

If grammaticality talk targets operator value, then grammaticality judgments are themselves a social technology: a way of policing the operator repertoire of a communicative situation. This aligns with two independently motivated ideas.

First, grammatical conventions are community property: they're the stabilized solutions that allow rapid coordination under uncertainty \citep{clarkbrennan1991,stalnaker2002}. \textcite{oconnor2019} demonstrates that such coordination problems drive communities to converge on categorical signals that effectively solve local problems but readily entrench arbitrary exclusions. This coordination story shouldn't be confused with naive instrumentalism. As \textcite{pullum2019} emphasises, speakers don't have to get grammar right in order to be understood~-- there's very little pressure to comply for strictly communicative reasons.\footnote{This empirical fact has not noticeably reduced the market for usage advice.} The relevant commitment is practice-constitutive: participating in a linguistic practice incurs a commitment to its constraints \citep{millar2004}, but this doesn't entail moral obligation, and the accountability is indexed to expected competence. Three variables should be distinguished here: the \term{operator repertoire} (which contrasts exist and what values they conventionally take), the \term{activation profile} (which contrasts are treated as \enquote{in play} in a given register, activity type, or genre), and the \term{accountability profile} (how strictly departures are treated given participant roles and expectations). The repertoire is relatively stable for a community. The activation profile varies: formal written registers may activate operators (e.g.\ subjunctive marking) that are dormant in casual speech. The accountability profile varies further: second-language learners, young children, and visitors to an unfamiliar dialect region participate in the same operator stratum~-- the conventions themselves don't change~-- but participant roles calibrate expectations. The operators remain publicly available; what shifts is which are active and how departures are treated.

Second, the repair system treats some departures as requiring correction and others as negotiable \citep{schegloffjeffersonsacks1977}. Operator violations are predicted to sit at the intersection: they're departures that threaten coordination precisely because they disrupt publicly accountable control settings.

This framing also clarifies why categoricality is common but not inevitable. Some operator systems are heavily conventionalized, high-opportunity, and tightly constrained, which makes non-licensing stable and judgments sharp. Others are lower-opportunity or contested across overlapping norm centres, which makes judgments more variable. The key claim is that \emph{wherever categoricality emerges}, it should cluster around operator value.

A common objection is that speakers frequently label accent or dialect choices as \enquote{bad grammar} to enforce social hierarchies. Acknowledging the operator stratum doesn't require denying this reality. Instead, it provides a way to distinguish first-order interactional crashes from second-order ideological policing. Overt correction is rare in either case, but the two types differ in what they target. An operator mismatch targets the utterance and crashes the update: hearers fail to track reference or lack the epistemic anchoring the utterance presupposes. An ideological mismatch targets the speaker and leaves the update intact: the utterance is understood, but the reaction targets the person's social index \citep{agha2007,eckert2012}. One rejects the tool; the other rejects the hand that holds it. The fact that ideologies misappropriate the term \enquote{grammar} for the latter doesn't dissolve the reality of the former.

Similarly, lexical choices like slurs or taboo words are \enquote{publicly accountable} in a strong sense~-- using them can trigger immediate sanction~-- but they aren't operator violations. They don't typically cause the structural update to fail or the role allocation to crash; rather, they perform a move that's morally or socially impermissible. They're accountable as \emph{actions}, whereas operator violations are accountable as \emph{defective tools}. The \enquote{you can't say that} reaction to a slur is a blocking of the social move; the \enquote{you can't say that} reaction to an operator failure is a rejection of the instrument.

A caveat on scope: this paper addresses what triggers grammaticality talk, not how such talk is socially deployed. The operator stratum provides the material that language ideology works on; it doesn't follow that operator structure is ideology-free or that judgments are politically neutral. Who speaks for \enquote{the community}? Who gets to police the repertoire, and with what consequences for whom? These are real questions, but they're questions about the social life of grammaticality talk, not about its target. That's a separate inquiry.

% \section{Integration with the MVMG architecture}
%
% This account aligns directly with the formal model of grammaticality developed in the companion paper (Reynolds, \textit{Grammaticality De-idealized}). That model defines the objective stability of a grammatical norm $G(u)$ as the product of situational licensing $C_t(u)$, interpretive coherence $K(u)$, and structural viability \textsf{map}. The operator stratum hypothesis specifies the domain of $C_t$.
%
% In the MVMG framework, $C_t(u)$ tracks whether a form is licensed in the current communicative situation. The central claim here is that $C_t$ tracks \emph{operator values}. When a speaker produces a form that violates an operator constraint (e.g., a switch-reference error or a tense mismatch), it drives $C_t \to 0$, resulting in a catastrophic drop in $G(u)$. This corresponds to the categorical \enquote{ungrammatical} judgment.
%
% By contrast, indexical mismatches (accent, register shifts) don't target $C_t$. They may lower the subjective feeling of grammaticality $F(u)$ via social penalty terms ($P_{other}$), or they may affect interpretive coherence $K(u)$ if the signal is hard to process, but they don't zero out the licensing term. This separation explains why operator violations feel effectively \enquote{unnegotiable} (the licensing term is binary or near-binary) while indexical mismatches feel gradient or situational. The \enquote{high opportunity mass} diagnostic explains how $C_t$ stabilizes: frequent mandatory updating of operator values provides the dense data needed for the community to converge on sharp non-licensing boundaries.

\section{Predictions and research strategies}

The proposal generates tractable empirical predictions across subdisciplines, which is part of its point: it's meant to be usable by typologists, sociolinguists, psycholinguists, and interaction analysts.

\textsc{(1) Repair asymmetries.} Overt repair is rare overall, but when it does occur, operator mismatches should show a different profile from indexical mismatches: more likely to be formulated as incomprehension or rejection (\enquote{what?}, \enquote{who did it?}) than as negotiation of stance or appropriateness \citep{schegloffjeffersonsacks1977}. Cross-linguistically, the targets of repair should correlate with what is grammaticalized. Operationally: in annotated conversational corpora, code operator mismatches (tense errors, agreement failures, switch-reference violations) separately from indexical mismatches (accent shifts, register mismatches); the prediction is that operator mismatches show higher rates of open-class repair initiation and explicit rejection, while indexical mismatches show more stance negotiation and accommodation. A caveat: many apparent \enquote{operator errors} in naturalistic data are dialect differences or L2 phenomena, and repair is sensitive to participant roles. The prediction is conditional on controlling for these factors or~-- more usefully~-- treating them as part of the accountability profile theorised in \S8.

\textsc{(2) Satiation asymmetries.} Repetition should attenuate anomaly responses more readily for non-operator mismatches and for low-entrenchment operator candidates than for high-opportunity, strongly preempted operator gaps. This follows from the role of opportunity mass in stabilizing non-licensing \citep{bybee2006,bybee2010}. Operationally: in acceptability judgment tasks with repeated exposure, high-opportunity operator violations (e.g., subject-verb agreement in obligatory contexts) should resist satiation, while pragmatic oddities and low-frequency operator mismatches should show steeper improvement curves.

\textsc{(3) Phonological operator effects.} In languages where tone or intonation functions as an operator exponent, violations should elicit judgment and processing profiles more similar to morphosyntactic anomalies than to accent shifts, consistent with the idea that the relevant factor is operator value, not segmental content \citep{hyman2011,ladd2008}. Operationally: compare ERP responses to tonal tense violations, segmental agreement violations, and sociophonetic accent shifts; the prediction is that the first two pattern together (e.g., P600-like profiles typically associated with morphosyntactic anomaly) while the third diverges.

\textsc{(4) Sociolinguistic stratification without grammaticalization.} Accent and register differences should show systematic indexical fields and enregisterment dynamics \citep{eckert2008,agha2007}, but the metalinguistic categorization as \enquote{grammar} should be predicted to arise mainly when the indexical resource becomes part of a closed operator system (e.g.\ a grammaticalized honorific agreement paradigm, not merely polite lexical choice).

\textsc{(5) Processing signatures as responses to operator failure.} ERP and related measures should distinguish operator mismatches from conceptual anomalies in ways that are consistent with their different roles in public update, without requiring a modular syntax/semantics division \citep{kutashillyard1980,osterhoutholcomb1992,friederici2011}.

None of these predictions presupposes a particular formal architecture. They presuppose only that communities conventionalize a subset of contrasts as operator settings for public coordination.

\section{Conclusion}

The motivating puzzle was why clause structure is treated as \enquote{grammar} in much the same way as tense or number marking is, while accent and much lexical choice are treated differently. The proposed answer is that grammaticality talk tracks a difference in value, and that this difference reflects a three-layer organization of linguistic infrastructure. Communities conventionalize expression-shape constraints (phonotactics, morphotactics) that make utterances recognisable as tokens of the system; \term{operators}~-- closed-paradigm contrasts that configure public update, allocate participant roles, and constrain uptake; and payload resources (open-class lexicon, indexical field) that are negotiable and extensible. Grammaticality talk targets the middle layer.

Clause structure belongs to the operator stratum because it packages operators: clause typing, argument linking, dependency management, and cross-clausal coherence. Tense and number belong where they're grammaticalized as operators. Accent and open-class lexical choice typically don't, because they primarily contribute indexical and conceptual value whose accountability regime is different~-- though phonological and gestural material can enter the operator stratum when grammaticalized.

An information-theoretic perspective clarifies why operators attract categorical policing. Operators function as protocol headers rather than payload content: they carry few bits in themselves but cause large entropy reduction downstream, constraining the space of licit interpretations and next moves. A wrong operator value doesn't merely produce a surprising concept combination; it disrupts the mapping from form to public update. This explains why operator mismatches feel like a different kind of wrongness from infelicitous lexical choices or unexpected accent.

This reframing makes three payoffs explicit. It de-Englishes the problem by predicting that \enquote{what counts as grammar} varies with which contrasts are grammaticalized, while preserving the generalization that grammaticality targets operator value. It makes room for phonological and gestural operator exponents without weakening the empirical distinction between operator mismatch and style mismatch. And it yields a research programme with converging evidence streams: typological distribution, processing signatures, and preemption dynamics.

If grammaticality names stability in the operator repertoire, then the special status of clause structure isn't evidence for autonomous syntax~-- it's evidence that some form--value relations function as infrastructure. The asterisk marks where that infrastructure is absent; the judgment marks where the community is present.

\newpage
\printbibliography
\end{document}