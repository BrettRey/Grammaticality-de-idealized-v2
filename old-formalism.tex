\section{A Simple Formal Model of the Framework}\label{sec:formalism}

This simple, descriptive formal model illustrates how morphosyntactic structure, meaning, and community norms interact to determine grammaticality. For any utterance $u$, let:

\begin{enumerate}
    \item $M(u)$ be the morphosyntactic structure of $u$. $M(u)$ encompasses word order, inflection, and syntactic relationships. It attempts to generate a morphosyntactic meaning, $\mu(u)$.
    \item $\mu(u)$ be the morphosyntactic meaning of $u$, derived from the structure $M(u)$. This represents the meaning contributed specifically by the morphosyntactic form.
    \item $\sigma(u)$ be the composite socio-semantic--pragmatic meaning of the utterance, including phonological, lexical, and constructional components. This represents the overall intended or contextually derived meaning, integrating morphosyntactic, lexical, and pragmatic information.
    \item $K(u) \in [0,1]$ be a compatibility function measuring coherence between $\mu(u)$ and $\sigma(u)$. $K(u) = 1$ indicates full coherence, meaning that $\mu(u)$ provides a valid structural basis for $\sigma(u)$. $K(u) = 0$ indicates a fundamental clash, where $\mu(u)$ provides a structure that is incompatible with the intended or contextually relevant meaning in $\sigma(u)$. This clash can arise from conflicting temporal specifications, incompatible argument structures, or other mismatches between the structural implications of the morphosyntax and the overall meaning.
    \item $C^t(u) \in [0,1]$ be the community acceptance of the form--meaning pairing, $M(u)$--$\mu(u)$, at time $t$. $C^t(u) = 1$ indicates full acceptance (the pairing of $M(u)$ and $\mu(u)$ is conventional and widely used). $C^t(u) = 0$ indicates no acceptance (the pairing isn't recognised or is actively rejected). Values between 0 and 1 represent varying degrees of acceptance, reflecting factors like frequency, regional variation, and ongoing language change.
    \item $F(u) \in [-1,0]$ be the \enquote{feeling of ungrammaticality}, where $-1$ indicates strong rejection and $0$ indicates no negative feeling. $F(u)$ reflects the subjective response to perceived instability in the form--meaning relationship and can be influenced by factors like processing difficulty, the strength of the violated convention (reflected in $C^t(u)$), and the type of mismatch between $\mu(u)$ and $\sigma(u)$.
\end{enumerate}

The grammaticality $G(u)$ of an utterance at time $t$ is then defined as:

\[
G(u) = C^t(u) \cdot K(u) \cdot \begin{cases}
1 & \text{if } M(u) \text{ evokes a } \mu(u) \\
0 & \text{otherwise}
\end{cases}
\]

\noindent \textbf{Interpretation Rule:} In the formula for $G(u)$, if $M(u)$ doesn't evoke a $\mu(u)$, the conditional part evaluates to 0, resulting in $G(u) = 0$. Although $K(u)$ and $C^t(u)$ are technically undefined in this situation (as there is no $\mu(u)$ to assess for compatibility or community acceptance), we interpret them as having a value of 0 for the purposes of calculating $G(u)$. This interpretation reflects the fundamental principle of MMMG: if the morphosyntactic form fails to generate any meaning, the entire form--meaning relationship is disrupted, and grammaticality is zero, regardless of other factors.

\subsection{Applying the Model}

Here I illustrate applications of the model to the following cases from previous sections.

\begin{enumerate}
    \item \textit{Colorless green ideas sleep furiously}:
    \begin{itemize}
        \item \(M(u)\): Declarative clause with bare plural subject and intransitive predicate evokes \(\mu(u)\).
        \item \(\mu(u)\): Generic plural subject performing action with manner modification.
        \item \(\therefore \begin{cases} 1 \end{cases}\) (condition 1: \(M(u)\) yields a \(\mu(u)\)).
        \item \(\sigma(u)\): Abstract concepts with contradictory properties performing physically impossible actions in an intense manner.
        \item \(K(u) = 1\) (no conflict between \(\mu(u)\) and \(\sigma(u)\)).
        \item \(C^t(u) = 1\) (The community accepts the pairing of \(M(u)\) with \(\mu(u)\)).
        \item \(\therefore G(u) = 1 \cdot 1 \cdot 1 = 1\) (fully grammatical).
        \item \(F(u) \approx 0\) (Minimal negative feeling despite semantic oddity, as there's no morphosyntactic violation).
    \end{itemize}

    \item \textit{Can the have running?}:
    \begin{itemize}
        \item \(M(u)\): list of words fails to evoke \(\mu(u)\).
        \item \(\mu(u)\): No coherent morphosyntactic meaning.
        \item \(\therefore \begin{cases} 0 \end{cases}\) (condition 1: \(M(u)\) fails to yield a \(\mu(u)\)).
        \item \(\sigma(u)\): Question about ability or permission and a running activity.
        \item \(K(u) = 0\) (By the interpretation rule, since \(M(u)\) fails to evoke \(\mu(u)\), \(K(u)\) is considered 0).
        \item \(C^t(u) = 0\) (By the interpretation rule, since \(M(u)\) fails to evoke \(\mu(u)\), \(C^t(u)\) is considered 0).
        \item \(\therefore G(u) = 0 \cdot 0 \cdot 0 = 0\) (categorically ungrammatical).
        \item \(F(u) = -1\) (Immediate strong negative feeling due to the complete failure of form--meaning mapping).
    \end{itemize}

    \item \textit{I've finished it yesterday}:
    \begin{itemize}
        \item \(M(u)\): Declarative clause evokes \(\mu(u)\).
        \item \(\mu(u)\): First-person subject has completed an action with relevance to the present with temporal modification.
        \item \(\therefore \begin{cases} 1 \end{cases}\) (condition 1: \(M(u)\) yields a \(\mu(u)\)).
        \item \(\sigma(u)\): Speaker indicates completion of an action during the previous day, with the intended meaning focusing on the finished state of the action rather than its current relevance.
        \item \(K(u) = 0\) (Nearly complete incompatibility due to the temporal clash between present perfect structure suggesting current relevance and \textit{yesterday} indicating completed action with no current relevance).
        \item \(C^t(u) = 1\) (The community accepts the pairing of \(M(u)\) with \(\mu(u)\)).
        \item \(\therefore G(u) = 1 \cdot 0 \cdot 1 = 0\) (ungrammatical due to meaning clash).
        \item \(F(u) = -1\) (Strong negative feeling due to the temporal clash).
    \end{itemize}

    \item \textit{It very good} (expecting \textit{it's very good}):
    \begin{itemize}
        \item \(M(u)\): Pronoun + AdjP evokes \(\mu(u)\).
        \item \(\mu(u)\): Predication of quality to a subject.
        \item \(\therefore \begin{cases} 1 \end{cases}\) (condition 1: \(M(u)\) yields a \(\mu(u)\)).
        \item \(\sigma(u)\): Speaker assesses an object or situation positively, in an informal or non-native English context.
        \item \(K(u) = 1\) (\(\mu(u)\) and \(\sigma(u)\) align).
        \item \(C^t(u) = 0.1\) (The community doesn't accept this \(M(u)\) for \(\mu(u)\) at $t$, though it's not entirely unfamiliar).
        \item \(\therefore G(u) = 0.1 \cdot 1 \cdot 1 = 0\) (ungrammatical despite meaning coherence).
        \item \(F(u) = -0.9\) (Strong negative feeling due to the missing copula).
    \end{itemize}

    \item \textit{It very good} (expecting \textit{I consider it very good}):
    \begin{itemize}
    \item \(M(u)\): Pronoun + AdjP evokes \(\mu(u)\).
    \item \(\mu(u)\): Predication of quality to a subject.
    \item \(\therefore \begin{cases} 1 \end{cases}\) (condition 1: \(M(u)\) yields a \(\mu(u)\)).
    \item \(\sigma(u)\): Speaker evaluates something positively, in a context where the utterance is expected to be part of a larger evaluative construction.
    \item \(K(u) \approx 0.5\) (Partial compatibility as \(\mu(u)\) provides only part of the structure needed for the intended meaning).
    \item \(C^t(u) \approx 0.7\) (In \enquote{small clauses} and certain complement structures, this form is acceptable in standard English).
    \item \(\therefore G(u) \approx 0.7 \cdot 0.5 \cdot 1 \approx 0.35\) (moderate grammaticality).
    \item \(F(u) \approx -0.3\) (Moderate negative feeling due to apparent incompleteness in this context).
    \end{itemize}

    \item \textit{We sheared three sheeps}:
    \begin{itemize}
        \item \(M(u)\): Declarative clause with subject, verb, and quantified object evokes \(\mu(u)\).
        \item \(\mu(u)\): A first-person group performed a completed action on a specific quantity of animals.
        \item \(\therefore \begin{cases} 1 \end{cases}\) (condition 1: \(M(u)\) yields a \(\mu(u)\)).
        \item \(\sigma(u)\): Speaker reports that they and others removed wool from three individual ovine animals.
        \item \(K(u) = 1\) (\(\mu(u)\) and \(\sigma(u)\) align).
        \item \(C^t(u) = 0.1\) (The community doesn't accept this \(M(u)\) for \(\mu(u)\); the zero-marked form is required).
        \item \(\therefore G(u) \approx 0.1 \cdot 1 \cdot 1 \approx 0.1\) (very low grammaticality).
        \item \(F(u) = -1\) (Strong negative feeling due to violation of irregular plural formation rule).
    \end{itemize}

    \item \textit{I saw Joan, a friend of whose was visiting}:
    \begin{itemize}
        \item \(M(u)\): Declarative clause with object NP + adjunct relative clause containing independent relative \textit{whose} evokes \(\mu(u)\).
        \item \(\mu(u)\): First-person subject completed an action involving a person connected to another entity via possessive relation.
        \item \(\therefore \begin{cases} 1 \end{cases}\) (condition 1: \(M(u)\) yields a \(\mu(u)\)).
        \item \(\sigma(u)\): Speaker reports seeing Joan, and additionally notes that someone connected to Joan was visiting.
        \item \(K(u) = 1\) (\(\mu(u)\) and \(\sigma(u)\) align).
        \item \(C^t(u) \approx 0.3\) (Independent relative \textit{whose} is rare and many speakers would prefer \textit{whose friend}).
        \item \(\therefore G(u) \approx 0.3 \cdot 1 \cdot 1 = 0.3\) (low to moderate grammaticality).
        \item \(F(u) \approx -0.7\) (Strong negative feeling for many speakers due to unfamiliarity with and difficulty processing the independent relative \textit{whose}).
    \end{itemize}
\end{enumerate}

This analysis shows how ungrammaticality can arise through distinct paths while maintaining a unified formal framework. A construction can be ungrammatical because: (i) its morphosyntactic form fails to evoke any meaning (\(M(u)\) fails to yield \(\mu(u)\), as in \textit{Can the have running?}), (ii) its morphosyntactic meaning clashes with its composite meaning (\(K(u) = 0\), as in \textit{I've finished it yesterday} or the independent use of \textit{It very good}), or (iii) the community doesn't (yet) accept this form for expressing this meaning (\(C^t(u) = 0\), as in \textit{sheeps} or predicative \textit{It very good}).  Community acceptance isn't static: a construction like \textit{I've finished it yesterday}~-- currently ungrammatical due to its \(K(u) = 0\)~-- could become grammatical if diachronic changes in usage lead speakers to reinterpret the tense-aspect system, thereby altering the compatibility between \(\mu(u)\) and \(\sigma(u)\).

Some cases, like \textit{It very good}, can be ungrammatical in different ways depending on what meaning the speaker is attempting to express. The framework also captures gradient acceptability through partial community acceptance (\(C^t(u) \approx 0.3\) for independent \textit{whose}). Both categorical and gradient ungrammaticality emerge from the same basic components. From this perspective, grammaticality is a dynamic equilibrium, shaped by the interaction of structural coherence and communal negotiation.

\subsection{Community Dynamics and Actuation}

The evolution of community acceptance $C^t(u)$ follows Verhulst's logistic equation from population dynamics (see \textcite{kauhanen2025} for a lucid linguistic introduction):
\[
\frac{dC}{dt} = \Delta \cdot C(1 - C)
\]
where $\Delta = \alpha - \beta$ represents the net bias towards acceptance. Here, $\alpha$ denotes the rate at which speakers switch from rejection to acceptance upon exposure to the construction, whilst $\beta$ represents the opposing flow from acceptance back to rejection. This differential equation yields the familiar S-curve of language change~\autocite{kroch1989}, emerging naturally from the microscopic dynamics of speaker interactions rather than being imposed as a descriptive convenience.

The parameter $\Delta$ serves as a stability criterion for grammatical change. When $\Delta > 0$, the construction enjoys a net advantage, and community acceptance grows logistically towards full adoption. When $\Delta < 0$, the form faces systematic disadvantage, and acceptance declines towards zero. The critical moment of \textit{actuation}~-- when a previously marginal construction begins to spread through the community~-- occurs precisely when $\Delta$ crosses zero, transforming the equilibrium at $C = 0$ from stable to unstable.

This bifurcation framework clarifies why certain innovations succeed whilst others fail: successful actuation requires not merely the presence of innovative speakers, but a fundamental shift in the community's linguistic dynamics such that $\alpha$ exceeds $\beta$. Such shifts may arise from language contact, generational change, processing pressures, or interactions with other grammatical subsystems.

For small speech communities ($N < 10^4$), stochastic effects become significant, and the deterministic logistic growth must be supplemented with fixation probabilities derived from birth--death processes~\autocite{kauhanen2022,nowak2006}. In such cases, even constructions with $\Delta > 0$ may fail to achieve community acceptance due to random extinction events, whilst those with $\Delta < 0$ may occasionally succeed through demographic accidents.\footnote{The exact fixation probability for $k$ initial adopters in a population of $N$ speakers is given by the formula in \textcite{nowak2006}, but for most purposes the deterministic approximation suffices when $N$ is large.}

This formulation captures several key aspects of the framework:

\begin{itemize}
    \item A morphosyntactic structure must evoke a meaning to be potentially grammatical.
    \item The composite and morphosyntactic meanings must be compatible to some degree.
    \item Community acceptance follows principled dynamics rather than arbitrary curves.
    \item Subjective feelings of ungrammaticality ($F(u)$) are distinct from objective grammaticality ($G(u)$).
    \item Actuation can be defined precisely as the moment when community dynamics favour the innovation.
\end{itemize}